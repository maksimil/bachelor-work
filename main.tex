\maketitlepages{Характеристические точки кривых ПЦР}

\section{Введение}

ПЦР в реальном времени --- распространённый метод для количественного анализа
ДНК образцов. Он заключается в наблюдении с помощью флуоресцентных красителей
за накоплением продукта в ходе циклов ПЦР
\cite{kubistaRealtimePolymeraseChain2006}.

Время реакции, когда сигнал от накопленной целевой ДНК слабее шума, называют
{\it фазой шума}. Предполагают, что, пока реакция имеет ресурсы, число
целевых молекул ДНК возрастает с каждым циклом в $(1+E)$ раз. Значение
$E\in[0,1]$ называют {\it эффективностью реакции}, это число полагают
постоянным до некоторого цикла. Время реакции, когда было накоплено достаточно
ДНК, чтобы выйти из фазы шума, но ресурсов ещё достаточно для накопления
продуктов ПЦР с постоянной эффективностью, называют {\it экспониенциальной
фазой}. Время реакции, когда ресурсов не достаточно для такого
экспониенциального роста, называют {\it фазой плато}
\cite{rebrikovRealtimePCRReview2006}.

В сделанных предположениях реакцию ПЦР до фазы плато можно описать законом
\[
  X_{n}=(1+E)^{n}X_0,
\]
где $X_0$ --- исходное количество целевых молекул ДНК в образце, $X_{n}$ ---
их количество после $n$-го цикла ПЦР.

Рассмотрим задачу определения отношения количеств целевых молекул ДНК между
образцами $A$ и $B$ с помощью ПЦР. Будем обозначать характеристики реакций
через соответствующие верхние индексы. Пусть мы смогли определить эффективности
реакций $A$ и $B$. Пусть также мы смогли определить (возможно дробные)
значения циклов $C_{p}^{(A)}, C_{p}^{(B)}$, на которых реакции $A$ и $B$
содержали одинаковое число молекул целевой ДНК
\cite{rasmussenQuantificationLightCycler2001}. Тогда
\[
  X_0^{(A)}(1+E^{(A)})^{C_{p}^{(A)}}=
  X_0^{(B)}(1+E^{(B)})^{C_{p}^{(B)}},
\]
и получим формулу
\[
  \frac{X_0^{(A)}}{X_0^{(B)}}=
  \frac{(1+E^{(B)})^{C_{p}^{(B)}}}{(1+E^{(A)})^{C_{p}^{(A)}}},
\]
которую можно использовать для нахождения относительного числа целевых молекул
ДНК в образцах. Точку $C_{p}$ называют {\it характеристической точкой} реакции
\cite{rebrikovPCRRealnomVremeni2009}.

Обозначим наблюдаемый флуоресцентный сигнал как $F_{n}$. Он зависит
от $X_{n}$ как
\begin{equation}\label{eq:signal}
  F_{n}=\alpha X_{n} + f_0+nf_1,
\end{equation}
где $f_0+nf_1$ --- {\it линейный фон} реакции
\cite{peccoudStatisticalEstimationsPCR1998}, $\alpha$ --- количество
флуоресцентного сигнала на единицу целевой ДНК
\cite{rebrikovRealtimePCRReview2006}. Значения $\alpha$ могут отличаться между
реакциями даже в пределах одного эксперимента
\cite{rebrikovRealtimePCRReview2006,larionovStandardCurveBased2005}. Обозначим
сигнал $F_{n}$ с вычтенным фоном как $\hat F_{n}=F_{n}-(f_0+nf_1)$. Обозначим
число циклов ПЦР как $N$.

  {\bf Постановка задачи:} Разработка, изучение и реализация методов
количественного анализа ДНК с помощью ПЦР.

\section{Оценка эффективности}\label{sec:eff}

Существует несколько методов нахождения эффективности реакции ПЦР. Например,
\begin{labeldesc}
  \item[$(\ln\hat F_{n})$]\label{item:method_log}
  Линейной регрессией по $\ln\hat F_{n}$ в экспониенциальной фазе на
  основании $\hat F_{n}=\alpha X_0(1+E)^{n}$ и
  \[
    \ln \hat F_{n}=\ln\alpha + \ln X_0+n\ln(1+E).
  \]
  Такой метод использован в
  \cite{wiesnerCountingTargetMolecules1992,ramakersAssumptionfreeAnalysisQuantitative2003}.
  % Однако он требует выделения $\hat F_{n}$ из $F_{n}$ и определения начала и
  % конца экспониенциальной фазы.

  \item[$(F_{n})$]\label{item:method_lin}
  Нелинейной регрессией по $F_{n}$ в экспониенциальной фазе на основании
  формулы
  \[
    F_{n}=\alpha X_0(1+E)^{n}+f_0+nf_1.
  \]
  Данный метод позволяет учесть в модели линейный фон. Без члена $nf_1$ он
  использован в \cite{tichopadStandardizedDeterminationRealtime2003}. Без
  члена $nf_1$ и с фиксированным $f_0$ как средним значением первых пяти
  $F_{n}$ он использован в \cite{barKineticOutlierDetection2003}.

  \item[$(\hat F_{n_1}\hat F_{n_2})$]\label{item:method_sys}
  Нахождением $E$ из системы уравнений
  \[
    \hat F_{n_1}=\hat F_0(1+E)^{n_1},\qquad \hat F_{n_2}=\hat F_0(1+E)^{n_2},
  \]
  где $\hat F_{n_1},\hat F_{n_2},n_1,n_2$ известны, $\hat F_0,E$ неизвестны,
  циклы $n_1,n_2$ находятся в экспониенциальной фазе
  \cite{liuNewQuantitativeMethod2002}. В \cite{liuNewQuantitativeMethod2002}
  $n_1,n_2$ определены как первые циклы, на которых значение $\hat F_{n}$ стало
  выше некоторых пороговых значений.

  \item[$(\hat m_{n'})$]\label{item:method_estimate}
  Нахождением $E$ по оценке
  \[
    E=\hat m_{n'}-1,\qquad
    \hat m_{n}=\frac{\hat F_{n-1}+\hat F_{n}+\hat F_{n+1}}
    {\hat F_{n-2}+\hat F_{n-1}+\hat F_{n}},
  \]
  где $n'$ --- некоторый цикл, на котором
  $\delta_{n}=|\hat m_{n-1}-\hat m_{n}|$ достигает достаточно малого значения,
  например, $\delta_{n}<0.05$ \cite{peccoudStatisticalEstimationsPCR1998}.

  \item[$(10^{-k})$]\label{item:method_dilutions}
  По последовательным разбавлениям. Пусть мы подготовили образцы с
  $X_0^{(k)}=I\cdot 10^{-k}$. Тогда
  \[
    X'=X_0^{(k)}(1+E)^{C_{p}^{(k)}},
  \]
  \[
    \ln X' = -k\ln 10 + \ln I + C_{p}^{(k)}\ln (1+E),
  \]
  \[
    C_{p}^{(k)}=-\frac{\ln 10}{\ln (1+E)}k+\ln\frac{X'}{I}=:kc_1+c_2,
  \]
  где $c_1,c_2$ можно найти линейной регрессией и из $c_1$ найти $E$
  \cite{rasmussenQuantificationLightCycler2001}. Однако справедливость
  этого метода сильно зависит от предположения одинаковой эффективности
  образцов, которое далеко не всегда выполняется
  \cite{ramakersAssumptionfreeAnalysisQuantitative2003}.
\end{labeldesc}

Все методы работают с точками в экспониенциальной фазе. Метод
\ref{item:method_sys} требует определения двух точек в экспониенциальной фазе.
Методы \ref{item:method_log}, \ref{item:method_lin} требуют определения границ
экспониенциальной фазы. Метод \ref{item:method_dilutions} требует определения
характеристических точек реакций.

Методы \ref{item:method_log} и \ref{item:method_lin} требуют решения задачи
оптимизации, однако их целевые функции различаются:
\[
  \delta_{\text{\ref{item:method_log}}}=\frac{1}{2}\sum_{n=n_1}^{n_2}(\ln \beta_0+n\ln(1+E)-\hat F_{n})^{2},
\]
\[
  \delta_{\text{\ref{item:method_lin}}}=\frac{1}{2}\sum_{n=n_1}^{n_2}(\beta_0(1+E)^{n}+f_0+nf_1-F_{n})^{2},
\]
где $\beta_0,E,f_0,f_1$ --- параметры моделей.

Особенности методов представлены в таблице \ref{table:eff_analysis}.

\begin{table}
  {
    \centering
    \begin{tabular}{r|cccccccc}
                                  & ТЭФ      & ГЭФ      & $C_{p}$  & $\hat F_{n}$ & НЗ       & ЛР       & НР       \\\hline
      \ref{item:method_log}       &          & $\times$ &          & $\times$     &          & $\times$ &          \\
      \ref{item:method_lin}       &          & $\times$ &          &              &          &          & $\times$ \\
      \ref{item:method_sys}       & $\times$ &          &          & $\times$     &          &          &          \\
      \ref{item:method_estimate}  &          &          &          & $\times$     &          &          &          \\
      \ref{item:method_dilutions} &          &          & $\times$ &              & $\times$ & $\times$ &
    \end{tabular}
    \caption{Анализ методов определения эффективности.}
    \label{table:eff_analysis}
  }

  ЭФ --- экспониенциальная фаза, ТЭФ --- точки ЭФ, ГЭФ --- границы ЭФ, $C_{p}$
  --- определение $C_{p}$, $\hat F_{n}$ --- выделение $\hat F_{n}$ из
  $F_{n}$, НЗ --- несколько запусков ПЦР, ЛР --- линейная регрессия, НР ---
  нелинейная регрессия. Первая строка содержит требования. Если метод имеет
  требование, то соответствующая ячейка отмечена символом $\times$.
\end{table}

\section{Определение характеристической точки}\label{sec:cp}

Распространённый способ определения $C_{p}$ это дробный цикл, на котором
$\hat F_{n}$ достигает некоторого порогового значения, достигаемого пока
реакция ещё находится в экспониенциальной фазе
\cite{larionovStandardCurveBased2005,rasmussenQuantificationLightCycler2001}.
Такие методы называют {\it пороговыми}, в их контексте $C_{p}$ иногда
обозначают как $C_{t}$
\cite{kubistaRealtimePolymeraseChain2006,liuNewQuantitativeMethod2002}. Однако,
одинаковые значения сигнала будут достигаться в разных пробирках при достаточно
различающихся количествах молекул целевой ДНК, если значения $\alpha$ из
\eqref{eq:signal} достаточно различаются между запусками. Отсюда получили
распространение методы, основанные на определении $C_{p}$ по характеристикам
кривой, не зависящим от её умножения на постоянную
\cite{rebrikovRealtimePCRReview2006}. Назовём такие методы
{\it $\alpha$-инвариантными}.
% {\it (Замечание про нормализацию по
% амплитуде и про валидацию пороговых методов).}

Среди $\alpha$-инвариантных методов распространён выбор $C_{p}$ как максимум
второй производной по сглаженной кривой ПЦР
\cite{rasmussenQuantificationLightCycler2001}. Такой выбор делает следующие
предположения:
\begin{enumerate}[label=(\alph*)]
  \item{}Максимум второй производной достигается в экспониенциальной фазе
  реакции. Некоторые авторы вообще определяют конец экспониенциальной фазы
  в этой точке \cite{tichopadStandardizedDeterminationRealtime2003,
    zhaoComprehensiveAlgorithmQuantitative2005}.
  \item{}Максимум второй производной достигается во всех образцах при одном
  количестве накопленной целевой ДНК.\label{sdm_ass_2}
\end{enumerate}
Однако,
\begin{enumerate}[label=(\roman*)]
  \item{}Мы предполагаем, что в производных сглаженной динамики ПЦР есть
  смысл, не смотря на то, что ПЦР --- дискретный процесс.\label{sdm_note_1}
  \item{}Предположение \ref{sdm_ass_2} сложно обосновать физически, особенно с
  учётом замечания \ref{sdm_note_1}.\label{sdm_note_2}
\end{enumerate}
Несмотря на это данный метод получил распространение.

Сглаживание можно проводить приближением измеренных $F_{n}$ некоторой функцией
(моделью) методом наименьших квадратов. Будем рассматривать модели вида
\[
  M_{n}=f_0+nf_1+a\lbar M_{n}.
\]
Член $f_0+nf_1$ моделирует линейный фон, член $a\lbar M_{n}$ моделирует
динамику целевой ДНК. Например, в качестве $\lbar M_{n}$ можно взять
$(1+E)^{n}$ с параметром $E$, однако такая модель будет иметь смысл только до
фазы плато.

\subsection{Алгебраическая функция}

В
\cite{tichopadStandardizedDeterminationRealtime2003,zhaoComprehensiveAlgorithmQuantitative2005}
в качестве $\lbar M_{n}$ предложено использовать
\[
  A_{n}=\frac{1}{1+(n/n_0)^{-(b+1)}},
\]
где $b,n_0$ --- параметры, $b>0$. Такая модель, как и любые модели,
сглаживающие всю кривую ПЦР, учитывает затухание эффективности реакции, в
отличие от $\lbar M_{n}=(1+E)^{n}$.

Определим {\it эффективность в точке} как
$E^{(\lbar M)}_{n}=(\lbar M_{n+1}-\lbar M_{n})/\lbar M_{n}$. Для алгебраической
функции она выглядит как
\[
  E^{(A)}_{n}=\frac{A_{n+1}-A_{n}}{A_{n}}=\frac{A_{n+1}}{A_{n}}-1=
  \frac{1+(n/n_0)^{-(b+1)}}{1+((n+1)/n_0)^{-(b+1)}}-1.
\]
Динамика этого значения представлена на рис. \ref{fig:model_comparison}.
Заметим, что
\[
  \lim_{n\to 0+0}E^{(A)}_{n}=
  \lim_{n\to 0+0}\left(\frac{n}{n_0}\right)^{-(b+1)}=+\infty,
\]
что противоречит физическим представлениям о динамике реакции ПЦР.

\begin{figure}
  {\centering

    \begin{subfigure}{\textwidth}
      \centering
      \includegraphics[width=\textwidth]{./plots/pdf/plot1_a.pdf}
      \caption{График данных и моделей.}
    \end{subfigure}

    \begin{subfigure}{\textwidth}
      \centering
      \includegraphics[width=\textwidth]{./plots/pdf/plot1_b.pdf}
      \caption{График эффективностей.}
    \end{subfigure}

    \caption{Сравнение моделей алгебраической функции и сигмоиды.}
    \label{fig:model_comparison}

  }

  Рассмотрен {\tt FN, 4.SYBR} из
  \cite{karlenStatisticalSignificanceQuantitative2007}. (a) измеренные
  значения $F_{n}$ и полученные модели сигмоиды ($M_{n}^{(S)}$) и
  алгебраической функции ($M_{n}^{(A)}$). Для обеих моделей $R^{2}>0.9999$.
  (b) эффективности в точке для значений $\hat{F}_{n}$ ($f_0,f_1$ взяты из
  модели сигмоиды) и эффективности в точке моделей.

  При рассмотрении $E^{(S)}_{n}$ как модели $\hat F_{n+1}/\hat F_{n}-1$ имеем
  $R^{2}>0.98$, если рассматривать циклы начиная с 15 и $R^{2}>0.99$, если
  рассматривать циклы начиная с 20. При аналогичном рассмотрении
  алгебраическая модель доставляет $R^{2}\approx 0.70$ и $R^{2}\approx 0.96$
  соответственно.
\end{figure}

\subsection{Сигмоида}

В \cite{liuValidationQuantitativeMethod2002,
  rutledgeSigmoidalCurvefittingRedefines2004,
  liuProgressCurveAnalysis2011} предложено использовать
\[
  S_{n}=\frac{1}{1+e^{-p(n-n_0)}},
\]
где $p,n_0$ --- параметры, $p> 0$.

Её эффективность в точке выглядит как
\[
  \begin{aligned}
    E^{(S)}_{n}
     & =\frac{S_{n+1}-S_{n}}{S_{n}}=\frac{S_{n+1}}{S_{n}}-1= \\
     & =\frac{1+e^{-p(n-n_0)}}{1+e^{-p(n+1-n_0)}}-1=
    \frac{e^{-p(n-n_0)}-e^{-p(n+1-n_0)}}{1+e^{-p(n+1-n_0)}}=
    \frac{e^{p}-1}{1+e^{p(n+1-n_0)}}.
  \end{aligned}
\]
Динамика этого значения представлена на рис. \ref{fig:model_comparison}.
Заметим, что $E^{(S)}_{n}$ также является сигмоидой и
$\lim_{n\to+\infty}E^{(S)}_{n}=0$. Значение
\[
  E^{(S)}=\lim_{n\to-\infty}\frac{S_{n+1}-S_{n}}{S_{n}}=e^{p}-1
\]
можно интерпретировать как эффективность реакции
\cite{swillensRevisitingSigmoidalCurve2008}. Отсюда можно сделать
предположение, что $p\in [0,\ln 2]$.

Модель алгебраической функции не обладает такой приятной интерпретируемостью
параметра $b$, и, как уже было сказано, $E^{(A)}_{n}$ не является адекватной
моделью эффективности реакции ПЦР.

Модель сигмоиды была предложена в \cite{liuValidationQuantitativeMethod2002} и
усовершенствована в \cite{rutledgeSigmoidalCurvefittingRedefines2004}
исключением части точек плато, поскольку она плохо их приближает.

\subsection{Двойная сигмоида}

В
\cite{kurnikPCRElbowDetermination2007,kurnikPCRElbowDetermination2011}
предложено использовать \enquote{двойную сигмоиду}, то есть функцию вида
\[
  D_{n}=S_{n}S'_{n}=\frac{1}{1+e^{-p(n-n_0)}}~
  \frac{1}{1+e^{-p'(n-n_0')}},
\]
где $p,n_0,p',n_0'$ --- параметры, $p>0,p'>0$.

В \cite{kurnikPCRElbowDetermination2011} предложено использовать максимум
кривизны функции $f(n)=aD_{n}$ для определения $C_{p}$. Такой метод не является
$\alpha$\hyp{}инвариантным. Как видно из формулы для кривизны функции
$f(x)=\alpha y(x)$
\[
  \kappa(x)=\frac{f''(x)}{(1+f'(x)^{2})^{3/2}}=
  \frac{\alpha y''(x)}{(1+\alpha y'(x)^{2})^{3/2}},
\]
её максимум может сместиться при разных $\alpha$. Чтобы в этом убедиться,
достаточно рассмотреть $y(x)=e^{x}$. К такому методу определения $C_{p}$ также
применимо замечание \ref{sdm_note_2} к методу максимума второй производной
на странице \pageref{sdm_note_2}.

Нахождение максимума второй производной моделей сигмоиды и алгебраической
функции можно провести по достаточно простым формулам
\cite{zhaoComprehensiveAlgorithmQuantitative2005}. Нахождение максимума второй
производной модели двойной сигмоиды, как и её теоретическое исследование,
требует более громоздких вычислений и рассуждений.

\section{Исходная оценка параметров}\label{sec:initial}

Методы сглаживания с помощью модели требуют нелинейной регрессии. Она
представляет из себя (обычно невыпуклую) задачу на глобальную оптимизацию, что
может быть затруднительно из-за попадания в локальные минимумы. Эту проблему
можно решить выбором исходной оценки и выбором метода оптимизации.

В \cite{liuProgressCurveAnalysis2011, goudarRobustParameterEstimation2009}
предложено использовать вместо метода Левенберга\hyp{}Марквардта метод Нелдера-Мида
при использовании модели сигмоиды.

Выбор исходной оценки проводится эвристическими методами. Можно использовать
сразу несколько методов, получив некоторое количество исходных оценок и
выбрать из них только то число лучших, которое позволяют вычислительные
способности. Поскольку все методы приближают S-образную кривую, оценки
параметров одной модели можно перевести в оценки параметров другой модели.

Исходная оценка параметра, связанного с эффективностью реакции можно проводить
через методы оценки эффективности.

\subsection{По пустым реакциям}

В \cite{peccoudStatisticalEstimationsPCR1998} предложено измерять линейный фон
$f_0+nf_1$ с помощью линейной регрессии по измеренным $F_{n}$ во время
проведения циклов ПЦР на пробирках без ДНК образца. Полученные значения
$f_0,f_1$ можно использовать как исходные или как фиксированные известные
значения соответствующих параметров.

\subsection{По нескольким точкам}\label{ssec:patent}

В \cite{barKineticOutlierDetection2003} предложено оценивать $f_0$ как среднее
значение по первым пяти $F_{n}$. Однако, в статье при моделировании не учтён
член $nf_1$.

В \cite{liuProgressCurveAnalysis2011} предложено оценивать $a$ как
$\max\{\hat{F}_{n}\}$.

В \cite{kurnikPCRElbowDetermination2007,kurnikPCRElbowDetermination2011}
предложено определять три исходных набора параметров на основании трёх точек
реакции: третье наименьшее значение $F_{n}$, третье наибольшее значение
$F_{n}$, первая точка над средним значением $F_{n}$. Во всех наборах параметров
предложено выбрать исходное $f_0$ как третье наименьшее значение $F_{n}$,
исхнодное $f_1$ как $0.01$. Исходное $n_0$ для обеих сигмоид модели двойной
сигмоиды предложено выбрать как номер первого цикла со значением $F_{n}$ выше
среднего значения $F_{n}$. Исходные $p,p'$ предложено брать из фиксированного
списка.

Вместо среднего значения при выборе $n_0$ можно использовать значение
$F_{\text{mid}}=(\max\{F_{n}\}+\min\{F_{n}\})/2$. В приложении
\ref{app:opt_an} показаны исходные оценки для двух вариантов выбора $n_0$ и
четырёх фиксированных значений $p$ (значения не взяты из патента) в применении
к модели сигмоиды. Можно увидеть, что выбор $n_0$ по $F_{\text{mid}}$ (по
крайней мере в рассмотренных случаях) работает лучше.

В \cite{kurnikPCRElbowDetermination2011} также предложено выбирать тот
результат оптимизации, на котором
\[
  \sum_{n=0}^{N-1}|M_{n}-F_{n}|
\]
принимает минимальное значение, а не целевая функция задачи оптимизации.

\subsection{Использование линейной регрессии}\label{ssec:linreg_improve}

В \cite{liuProgressCurveAnalysis2011,goudarRobustParameterEstimation2009}
предложен следующий метод. Пусть мы уже имеем оценку для $a,f_0,f_1$.
Предполагая $\hat F_{n}=aS_{n}$, получим
\[
  \frac{a-\hat F_{n}}{\hat F_{n}}=\frac{1-S_{n}}{S_{n}}=e^{-p(n-n_0)},
\]
\begin{equation}\label{eq:sigmoid_param_linreg}
  \ln\frac{a-\hat F_{n}}{\hat F_{n}}=-pn + pn_0.
\end{equation}
значения $p,n_0$ теперь можно оценить линейной регрессией на основании
\eqref{eq:sigmoid_param_linreg}. На практите стоит исключать из рассмотрения
точки, где $\hat F_{n}=0$ или $(a-\hat F_{n})/\hat F_{n}\leq 0$, которые могут
возникнуть из-за неточных оценок фона или шума.

Аналогичную оценку можно провести и для модели алгебраической функции.
\[
  \frac{a-\hat F_{n}}{\hat F_{n}}=\frac{1-A_{n}}{A_{n}}=(n/n_0)^{-(b+1)},
\]
\[
  \ln\frac{a-\hat F_{n}}{\hat F_{n}}=-(b+1)\ln n+(b+1)\ln n_0.
\]

Важно, что, пользуясь такими оценками мы минимизируем
\[
  \frac{1}{2}\sum_{n=0}^{N-1}\left(\ln \frac{a-\hat F_{n}}{\hat F_{n}}-
  \ln\frac{1-\lbar M_{n}}{\lbar M_{n}}
  \right)^{2},
\]
а не целевую функцию задачи подбора параметров модели. Для компенсации этой
неточности можем добавить веса к членам суммы. Определим
\[
  \delta'=\frac{1}{2}\sum_{n=0}^{N-1}w_{n}
  \left[
    \ln \left(\frac{1}{\lbar M_{n}}-1\right)-
    \ln \left(\frac{a}{\hat F_{n}}-1\right)
    \right]^{2},
\]
\[
  \delta=\frac{1}{2}\sum_{n=0}^{N-1}(a\lbar M_{n}-\hat F_{n})^{2}.
\]
Пусть $t$ --- некоторый параметр модели $\lbar M_{n}$. Если $\delta'$ достигает
минимум, то $\pt_{t}\delta'=0$. Предположим, что в этой точке
$a\lbar{M}_{n}\approx\hat F_{n}$, выберем $w_{n}$ так, чтобы в ней
$\pt_{t}\delta'\approx\pt_{t}\delta$. Формально такой выбор можно описать как
\[
  w_{n}=\lim_{\lbar M_{n}\to \hat F_{n}/a}c_{n}/c'_{n},
\]
где $c_{n},c_{n}'$ определены из выражений
\[
  \pt_{t}\delta = \sum_{n=0}^{N-1}c_{n}\pt_{t}\lbar M_{n},\qquad
  \pt_{t}\delta' = \sum_{n=0}^{N-1}w_{n}c'_{n}\pt_{t}\lbar M_{n}.
\]
В таком случае
\[
  c_{n}=a(a\lbar M_{n}-\hat F_{n}),
\]
\[
  c'_{n}=\left[
    \ln \left(\frac{1}{\lbar M_{n}}-1\right)-
    \ln \left(\frac{a}{\hat F_{n}}-1\right)
    \right]\frac{\lbar M_{n}}{1-\lbar M_{n}}\left(-\frac{1}{\lbar M_{n}^{2}}\right),
\]
\[
  \begin{aligned}
    \ln \left(\frac{1}{\lbar M_{n}}-1\right)-
    \ln \left(\frac{a}{\hat F_{n}}-1\right)
     & =
    \ln
    \frac{(1-\lbar M_{n})\hat F_{n}}{\lbar M_{n}(a-\hat F_{n})}=
    \ln\frac{\hat F_{n}-\hat F_{n}\lbar M_{n}}{a\lbar M_{n}-\lbar M_{n}\hat F_{n}}=  \\
     & =\ln\left(1+\frac{\hat F_{n}-a\lbar M_{n}}{\lbar M_{n}(a-\hat F_{n})}\right),
  \end{aligned}
\]
\[
  \frac{c'_{n}}{c_{n}}=\frac{1}{-a(\hat F_{n}-a\lbar M_{n})}
  \ln\left(1+\frac{\hat F_{n}-a\lbar M_{n}}{\lbar M_{n}(a-\hat F_{n})}\right)
  \frac{\lbar M_{n}}{1-\lbar M_{n}}\left(-\frac{1}{\lbar M_{n}^{2}}\right),
\]
\[
  \begin{aligned}
    \frac{1}{w_{n}}
     & =\lim_{\lbar M_{n}\to\hat F_{n}/a}\frac{c'_{n}}{c_{n}}=
    \lim_{\lbar M_{n}\to\hat F_{n}/a}
    \frac{1}{-a\lbar M_{n}(a-\hat F_{n})}\frac{\lbar M_{n}}{1-\lbar M_{n}}
    \left(-\frac{1}{\lbar M_{n}^{2}}\right)=                   \\
     & =\frac{a^{2}}{\hat F_{n}^{2}(a-\hat F_{n})^{2}}.
  \end{aligned}
\]
Отсюда $w_{n}=\hat F_{n}^{2}(a-\hat F_{n})^{2}/a^{2}$.

Описанным выше методом можно попробовать улучшить оценку $p,n_0$. Через $p,n_0$
можно линейной регрессией оценить $f_0,f_1,a$, через них можно оценить $p,n_0$
по алгоритму выше. Результат такого процесса продемонстрирован на примерах в
приложении \ref{app:opt_an}.

\subsection{Использование аппроксимационно-оценочных критериев}\label{ssec:app-est}

В
\cite{orekhovQuasiDeterministicProcessesMonotonic2021,orekhovUnsupervisedMachineLearning2023}
описаны {\it аппроксимационно-оценочные критерии}, ставящие себе цель
формализовать эвристику \enquote{метода локтя}. Определим
\[
  \delta_{\mu,m}^{2}(\varphi) = \min_{\alpha,\beta}\frac{1}{2}
  \sum_{n=1}^{\mu}[\alpha\varphi_{n}+\beta - F_{m-\mu+n}]^{2}.
\]
Это значение описывает, насколько отрезок наблюдений $F_{m-\mu+1}$,
$F_{m-\mu+2}$, ..., $F_{m}$  \enquote{близок} к $\varphi_1,...,\varphi_{\mu}$.
Пусть $\ell_{n}=n$, а $\psi_{n}$ --- некоторая нелинейная функция от $n$
(например, $\psi_{n}=2^{n}$). Будем считать, что процесс $F_{n}$ перешёл от
линейного роста к нелинейному между точками $m-1$ и $m$, если
\[
  \delta^{2}_{\mu,m-1}(\ell)\leq\delta^{2}_{\mu,m-1}(\psi),\qquad
  \delta^{2}_{\mu,m}(\ell)>\delta^{2}_{\mu,m}(\psi).
\]
Заметим, что $\delta^{2}_{\mu,m}(\varphi)$ является квадратичной формой от
вектора отрезка наблюдений циклов от $m-\mu+1$ до $m$. Таким образом,
аппроксимационно\hyp{}оценочные критерии определяют точку перехода как точку,
где квадратичная форма
$\rho_{\mu,m}(\psi)=\delta^{2}_{\mu,m}(\ell)-\delta^{2}_{\mu,m}(\psi)$
меняет знак на положительный. На практике лучше останавливать поиск не на
$\rho_{\mu,m}(\psi)>0$, а на $\rho_{\mu,m}(\psi)>\eps$, где $\eps$ ---
некоторое небольшое пололжиетльное число, например $10^{-4}$.

Аппроксимационно\hyp{}оценочные критерии в
\cite{orekhovUnsupervisedMachineLearning2023} разработаны для определения двух
точек перегиба графика ПЦР. Пусть $\mathcal C,\mathcal P$ --- множества
кандидатов на первую и вторую точку перегиба соответственно. Определим
\[
  \lbar f_0= F_0,\qquad
  \lbar f_1(c,p)=\frac{1}{2}\left(\frac{F_{c-1}-F_0}{c-1}
  +\frac{F_{N-1}-F_{p+1}}{N-p-2}\right),
\]
\[
  \mathcal K^{+}=\left\{(c,p)\in\mathcal C\times\mathcal P\;\left|\;
  F_{n}- [\lbar f_1(c,p)n+\lbar f_0]>-\eps'\right.
  \right\},
\]
где $\eps'$ --- некоторое положительное число (например, $0.5$). Пусть
$(\lbar{c}, \lbar{p})$ --- пара кандидатов из $\mathcal K^{+}$, доставляющая
$\lbar f_1$ наибольшее значение. Оценим $f_0$ как $\lbar f_0$, а $f_1$ как
$\lbar{f}_1(\lbar c,\lbar p)$. Из оценки параметров $f_0,f_1$ можно получить
оценки параметров других моделей, основанные на $\hat F_{n}$.

В \cite{orekhovUnsupervisedMachineLearning2023} алгоритм разработан без
допусков $\eps,\eps'$, однако тогда он становится слишком чувствителен к шуму.
Вместо $\lbar f_0=F_0$ можно использовать другие оценки параметра $f_0$.
Например, учитывая определение $\mathcal K^{+}$, можно взять
$\lbar{f}_0=\min\{F_{n}\}$. Вместо оценок $(F_{c-1}-F_0)/(c-1)$ и
$(F_{N-1}-F_{p+1})/(N-p-2)$, формирующих $\lbar f_1$, можно использовать
линейные регрессии по $F_{n}$ на отрезках $[0,c-1]$ и $[p+1,N-1]$.

\section{Заключение}

В данной работе рассмотрены методы количественного анализа ДНК с помощью
ПЦР. Был проведён литературный обзор на темы оценки эффективности реакции,
определения ХТ (характеристической точки), моделирования динамики реакции,
оптимизации и оценки параметров модели.

В части \ref{sec:eff} рассмотрены и проанализированы методы определения
эффективности реакции ПЦР, предложенные в литературе.

В части \ref{sec:cp} описаны методы определения ХТ и подробно изучен вопрос
выбора модели для сглаживания динамики ПЦР с целью определения ХТ как максимум
второй производной.

В части \ref{sec:initial} рассмотрен вопрос исходного подбора параметров модели
динамики ПЦР до их оптимизации. В \ref{ssec:patent} предложено выбирать $n_0$
по $F_{\text{mid}}$, а не по среднему значению $F_{n}$. В
\ref{ssec:linreg_improve} предложена модификация известного алгоритма оценки
параметров $p,n_0$ добавлением весов к целевой функции. В \ref{ssec:app-est}
предложены различные модификации, направленные на уменьшение чувствительности
метода к шуму.

В приложении \ref{app:opt_an} графически представлена задача оптимизации
параметров модели сигмоиды и описанные и предложенные в \ref{ssec:patent} и
\ref{ssec:linreg_improve} методы исходной оценки параметров. В приложении также
предложена модификация экспониенциальной модели, избегающая слишком малых
значений параметров и допускающая удобную их оценку.

\newpage

\showbib{}

\newpage

\appendix{}

\section{Анализ оптимизационной задачи модели сигмоиды}\label{app:opt_an}

Иллюстрации приложения \ref{app:opt_an} содержат графические представления
функции
\begin{equation}\label{eq:opt_fun}
  L(n_0,p)=\frac{1}{\max\{F_{n}\}-\min\{F_{n}\}}\min_{f_0,f_1,a}
  \sum_{n=0}^{N-1}(M_{n}-F_{n})^{2}
\end{equation}
и исходных оценок параметров $p,n_0$ модели сигмоиды для различных измеренных
$F_{n}$.

На рис. \ref{fig:opt_cont}, \ref{fig:opt_cont_3} показаны достаточно
\enquote{приятные} для модели сигмоиды данные. На них есть локальные минимумы,
в которые есть риск попасть, но их можно избежать разумной оценкой исходных
значений параметров.

На рис. \ref{fig:opt_cont_2} показан случай, с которым модель сигмоиды
справляется хуже. Меньшее $R^{2}$ можно обосновать шумом в данных. Однако,
значение параметра $a>10^{11}$ говорит о том, что модель не адекватно
представляет данные. Заметим, что в рассматриваемом случае
\[
  E^{(S)}\approx E^{(S)}_0\approx 0.66,\qquad
  E^{(S)}_{0}-E^{(S)}_{N-2}\approx 7\cdot 10^{-14}.
\]
То есть эффективность за 40 циклов не успела упасть и процесс не достиг фазы
плато. Отсюда можно предположить, что модель с $\lbar M_{n}=e^{pn}$ справится с
данными лучше и без слишком больших параметров в оптимальной точке. В
\cite{zhaoComprehensiveAlgorithmQuantitative2005} показано, что такая модель
приближает экспониенциальную фазу гораздо лучше сигмоиды. В рассматриваемом
случае модель экспоненты также доставляет $R^{2}>0.982$, но на этот раз в
оптимальной точке $a\approx 7\cdot 10^{-11}$, что уже имеет физическое
обоснование.

Чтобы избежать проблем с представлением настолько малых чисел на ЭВМ, можно
использовать $\lbar M_{n}=e^{p(n-(N-1))}$. Тогда в оптимальной точке
$a\approx 0.03$. Такая модель также открывает возможность для оценки $a$ как
$\hat F_{N-1}$.

\begin{figure}
  {\centering

  \begin{subfigure}{\textwidth}
    \centering
    \includegraphics[width=\textwidth]{./plots/pdf/plot2_a.pdf}
    \caption{}
  \end{subfigure}

  \begin{subfigure}{\textwidth}
    \centering
    \includegraphics[width=\textwidth]{./plots/pdf/plot2_b.pdf}
    \caption{}
  \end{subfigure}

  \begin{subfigure}{0.49\textwidth}
    \centering
    \includegraphics[width=\textwidth]{./plots/pdf/plot2_c.pdf}
    \caption{}
  \end{subfigure}
  \hfill
  \begin{subfigure}{0.49\textwidth}
    \centering
    \includegraphics[width=\textwidth]{./plots/pdf/plot2_d.pdf}
    \caption{}
  \end{subfigure}

  \caption{Изучение задачи подбора параметров модели сигмоиды для {\tt
  FN, 4.SYBR} из \cite{karlenStatisticalSignificanceQuantitative2007}.}
  \label{fig:opt_cont}

  }

  (a) Наблюдаемый сигнал $F_{n}$ и модель сигмоиды $M^{(S)}_{n}$. (b-d) Линии
  уровня функции \eqref{eq:opt_fun}. $G$ --- минимум. $P,P'$ --- исходные
  оценки по методу, описанному в части \ref{ssec:patent}. $p$ взяты как $0.2$,
  $0.4$, $0.55$, $\ln 2$. Для точек $P$ значение $n_0$ выбрано по среднему
  значению $F_{n}$, для $P'$ --- по $F_{\text{mid}}$. $I,I'$ --- оценки,
  улучшенные по методу без веса из части \ref{ssec:linreg_improve} от точек
  $P,P'$ соответственно. $I_{w},I_{w}'$ --- оценки, улучшенные по методу с
  весом от точек $P,P'$ соответственно.

\end{figure}

\begin{figure}
  {\centering

  \begin{subfigure}{\textwidth}
    \centering
    \includegraphics[width=\textwidth]{./plots/pdf/plot3_a.pdf}
    \caption{}
  \end{subfigure}

  \begin{subfigure}{\textwidth}
    \centering
    \includegraphics[width=\textwidth]{./plots/pdf/plot3_b.pdf}
    \caption{}
  \end{subfigure}

  \begin{subfigure}{0.49\textwidth}
    \centering
    \includegraphics[width=\textwidth]{./plots/pdf/plot3_c.pdf}
    \caption{}
  \end{subfigure}
  \hfill
  \begin{subfigure}{0.49\textwidth}
    \centering
    \includegraphics[width=\textwidth]{./plots/pdf/plot3_d.pdf}
    \caption{}
  \end{subfigure}

  \caption{Изучение задачи подбора параметров модели сигмоиды для {\tt
  Perl, 3.SYBR} из \cite{karlenStatisticalSignificanceQuantitative2007}.}
  \label{fig:opt_cont_2}

  }

  Обозачения как в рис. \ref{fig:opt_cont}. Здесь $N=40$ и в точке $G$ параметр
  $a\approx 2.6\cdot 10^{11}$.
\end{figure}

\begin{figure}
  {\centering
  \begin{subfigure}{\textwidth}
    \centering
    \includegraphics[width=\textwidth]{./plots/pdf/plot4_a.pdf}
    \caption{}
  \end{subfigure}

  \begin{subfigure}{\textwidth}
    \centering
    \includegraphics[width=\textwidth]{./plots/pdf/plot4_b.pdf}
    \caption{}
  \end{subfigure}

  \begin{subfigure}{0.49\textwidth}
    \centering
    \includegraphics[width=\textwidth]{./plots/pdf/plot4_c.pdf}
    \caption{}
  \end{subfigure}
  \hfill
  \begin{subfigure}{0.49\textwidth}
    \centering
    \includegraphics[width=\textwidth]{./plots/pdf/plot4_d.pdf}
    \caption{}
  \end{subfigure}

  \caption{Изучение задачи подбора параметров модели сигмоиды для {\tt
  K1/K2, 4.17E+06, Run\#1, Rep\#4} из \cite{rutledgeSigmoidalCurvefittingRedefines2004}.}
  \label{fig:opt_cont_3}

  }

  Обозачения как в рис. \ref{fig:opt_cont}.
\end{figure}
